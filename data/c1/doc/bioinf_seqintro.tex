\documentclass[]{article}
\usepackage{lmodern}
\usepackage{amssymb,amsmath}
\usepackage{ifxetex,ifluatex}
\usepackage{fixltx2e} % provides \textsubscript
\ifnum 0\ifxetex 1\fi\ifluatex 1\fi=0 % if pdftex
  \usepackage[T1]{fontenc}
  \usepackage[utf8]{inputenc}
\else % if luatex or xelatex
  \ifxetex
    \usepackage{mathspec}
    \usepackage{xltxtra,xunicode}
  \else
    \usepackage{fontspec}
  \fi
  \defaultfontfeatures{Mapping=tex-text,Scale=MatchLowercase}
  \newcommand{\euro}{€}
\fi
% use upquote if available, for straight quotes in verbatim environments
\IfFileExists{upquote.sty}{\usepackage{upquote}}{}
% use microtype if available
\IfFileExists{microtype.sty}{\usepackage{microtype}}{}
\ifxetex
  \usepackage[setpagesize=false, % page size defined by xetex
              unicode=false, % unicode breaks when used with xetex
              xetex]{hyperref}
\else
  \usepackage[unicode=true]{hyperref}
\fi
\hypersetup{breaklinks=true,
            bookmarks=true,
            pdfauthor={},
            pdftitle={},
            colorlinks=true,
            citecolor=blue,
            urlcolor=blue,
            linkcolor=magenta,
            pdfborder={0 0 0}}
\urlstyle{same}  % don't use monospace font for urls
\setlength{\parindent}{0pt}
\setlength{\parskip}{6pt plus 2pt minus 1pt}
\setlength{\emergencystretch}{3em}  % prevent overfull lines
\setcounter{secnumdepth}{0}


\begin{document}

\title{Short Intro to Sequence Alignment and Analysis for Biologists}
\author{Sebastian Schmeier}
\date{20th May 2014}
\maketitle

\section{Short Intro to Sequence Alignment and Analysis for
Biologists}\label{short-intro-to-sequence-alignment-and-analysis-for-biologists}

\subsection{0. Preface}\label{preface}

This introduction is exactly what the title says \emph{``short''}. It
does not try to be an exhaustive course or collection of tools and
exercises. It is meant to give a brief overview of what is out there.
Many good tools have been omitted. The fault is mine alone. This
\emph{``resource''} if you want to call it that, is evolving, so expect
changes at any time.

\textbf{In \emph{bioinformatics}}, a sequence alignment is a way of
arranging the sequences of DNA, RNA, or protein to identify regions of
similarity that may be a consequence of functional, structural, or
evolutionary relationships between the sequences
{[}\href{http://www.cshlpress.com/default.tpl?cart=14004538673655488\&fromlink=T\&linkaction=full\&linksortby=oop_title\&--eqSKUdatarq=466}{1}{]}.

Sequence alignments can be generally divided into three classes:
pairwise, multiple, and database searches:

\begin{itemize}
\itemsep1pt\parskip0pt\parsep0pt
\item
  Pairwise is easy to do (computationally). Often used to align large
  sequences (comparative genomics).
\item
  Multiple is more complex, but is more widely used (e.g.~phylogenetics,
  comparative genomics).
\item
  Database searches -- very widespread and useful. Usually in the form
  of BLAST. We will deal with this first!
\end{itemize}

\subsection{1. Database sequence search}\label{database-sequence-search}

We will use the \href{http://blast.ncbi.nlm.nih.gov/}{BLAST} tool to
find a sequence of interest, e.g.~to identify what the sequence possibly
does.

\textbf{BLAST} stands for \textbf{B}asic \textbf{L}ocal
\textbf{A}lignment \textbf{S}equence \textbf{T}ool
{[}\href{http://www.ncbi.nlm.nih.gov/pubmed/?term=9254694}{2}{]}. It is
available online through NCBI (National Center for Biotechnology
Information). The basic idea behind BLAST is that it searches your input
sequence against a \emph{database of DNA sequences} to find matches.
Matches are returned with \emph{e-values} that reflect the likelihood
that the match is real (as opposed to a chance match). It works in
association with GenBank -- the NCBI database for DNA sequences.

Attention! It is very important if you are using BLAST in your research,
that you know how it works, what its limitations are, and that you use
the correct BLAST options and interpret the meaning of the results
correctly. If usage of BLAST is presented in your thesis, this knowledge
will be expected, and you will also be expected to interpret your
results correctly.

\subsubsection{1.1. Get a sequence}\label{get-a-sequence}

Go to \url{http://www.ncbi.nlm.nih.gov/} and search for a gene ( e.g.
\href{http://www.ncbi.nlm.nih.gov/gquery/?term=BRCA2}{BRCA2}). Now we
see plenty of different databases where our search term was found. All
of these provide different information about our search term.

We want the sequence of the gene. Thus, we need to go to the
\href{http://www.ncbi.nlm.nih.gov/nuccore}{nucleotide} database (e.g.
\href{http://www.ncbi.nlm.nih.gov/nuccore/?term=BRCA2}{BRCA2}). Choose
for example the human sequence of your gene in FASTA format (more
information on data-formats can be found
\href{http://compbio.massey.ac.nz/wiki/\#!bioinf_files.md}{here}). Copy
a part of the sequence (e.g.
\href{http://www.ncbi.nlm.nih.gov/nuccore/1161383?report=fasta}{BRCA2}).

\subsubsection{1.2. Find the right BLAST
tool}\label{find-the-right-blast-tool}

Several different flavours of
\href{http://blast.ncbi.nlm.nih.gov/}{BLAST} are available.

\begin{itemize}
\itemsep1pt\parskip0pt\parsep0pt
\item
  \href{http://blast.ncbi.nlm.nih.gov/Blast.cgi?PROGRAM=blastn\&PAGE_TYPE=BlastSearch\&LINK_LOC=blasthome}{nucleotide
  blast} - Search a nucleotide database using a nucleotide query
\item
  \href{http://blast.ncbi.nlm.nih.gov/Blast.cgi?PROGRAM=blastp\&PAGE_TYPE=BlastSearch\&LINK_LOC=blasthome}{protein
  blast} - Search protein database using a protein query
\item
  \href{http://blast.ncbi.nlm.nih.gov/Blast.cgi?PROGRAM=blastx\&PAGE_TYPE=BlastSearch\&LINK_LOC=blasthome}{blastx}
  - Search protein database using a translated nucleotide query
\item
  \href{http://blast.ncbi.nlm.nih.gov/Blast.cgi?PROGRAM=tblastn\&PAGE_TYPE=BlastSearch\&LINK_LOC=blasthome}{tblastn}
  -Search translated nucleotide database using a protein query
\item
  \href{http://blast.ncbi.nlm.nih.gov/Blast.cgi?PROGRAM=tblastx\&PAGE_TYPE=BlastSearch\&LINK_LOC=blasthome}{tblastx}
  - Search translated nucleotide database using a translated nucleotide
  query
\item
  \href{http://blast.ncbi.nlm.nih.gov/Blast.cgi?PAGE_TYPE=BlastSearch\&BLAST_SPEC=blast2seq\&LINK_LOC=align2seq}{Align}
  - Align two (or more) sequences using BLAST (bl2seq)
\end{itemize}

\subsubsection{1.3. Choosing the right
options/parameters}\label{choosing-the-right-optionsparameters}

Some of these tools mentioned above have options/parameters for the type
of \href{http://blast.ncbi.nlm.nih.gov/}{BLAST} you can perform. It is
important to know what they are and what situations to use the different
options. Plus there are various other tools -- have a look around. There
are tutorials on the website.

The parameters section of th
\href{http://blast.ncbi.nlm.nih.gov/}{BLAST} tool is very important.
Here we talk about the most important ones:

\begin{itemize}
\itemsep1pt\parskip0pt\parsep0pt
\item
  \textbf{Max target sequences} - Maximum number of aligned sequences to
  display (the actual number of alignments may be greater than this).
\item
  \textbf{Short queries} - Automatically adjust word size and other
  parameters to improve results for short queries.
\item
  \textbf{Expect threshold} - Expected number of chance matches in a
  random model. The default value (10) means that 10 such matches are
  expected to be found merely by chance.
\end{itemize}

Note! The Expect value (\emph{e-value}) is critical -- simply put, it is
the number of such matches we expect by random chance. However, you
should look into exactly what the \emph{e-value} is. The lower the
e-value, the greater the chance the match is real. More information is
available
\href{http://www.ncbi.nlm.nih.gov/BLAST/blastcgihelp.shtml\#expect}{here}
and a video tutorial is available
\href{https://www.youtube.com/watch?v=nO0wJgZRZJs}{here}.

\begin{itemize}
\itemsep1pt\parskip0pt\parsep0pt
\item
  \textbf{The word size} - The length (number of nucleotides) of the
  seed that initiates an alignment (More information
  \href{http://www.ncbi.nlm.nih.gov/BLAST/blastcgihelp.shtml\#wordsize}{here}).
\item
  \textbf{Match/Mismatch Scores} - Reward and penalty for matching and
  mismatching bases, which determine whether to align residues or not --
  this is the basic engine of the search (More information
  \href{http://www.ncbi.nlm.nih.gov/BLAST/blastcgihelp.shtml\#Reward-penalty}{here}).
\item
  \textbf{Gap Costs} - Cost to create and extend a gap in an alignment
  (More information
  \href{http://www.ncbi.nlm.nih.gov/BLAST/blastcgihelp.shtml\#Reward-penalty}{here}).
\end{itemize}

Finally we have various filters for filtering out low-complexity
sequences. You can just filter these out from just the initial alignment
seeding, but include them the full alignment search, as well as
customize what regions get masked.

\subsubsection{1.4. Interpreting the
results}\label{interpreting-the-results}

\begin{quote}
\emph{\textbf{Example 1}} 1. Use the nucleotide sequence of your gene
and play with BLAST. 2. Copy a fairly short chunk of around 20-30 bp and
``BLAST'' it. What is the result? 3. Copy a bigger chunk, maybe
100-200bp of the gene and ``BLAST'' it. What is the result? 4. Delete
some stretches of the bigger chunk and ``BLAST'' again. What is the
result? 5. Insert some random DNA into the sequence and ``BLAST'' again.
What is the result?
\end{quote}

When looking at the results from BLAST, you should know what the various
results mean. Although the \emph{e-value} is very important, there are
other result options that are also important, such as the \emph{\%
coverage} of the sequence. Familiarize yourselves with these. You can
check the sequences to see what they are, by clicking on them, and the
alignments are also presented. Remember that BLAST is a local alignment
tool. It will find as many matches between the query and the subject as
fulfill the search criteria. This means that it is important to look at
all the results, as often there are many highly similar results, so to
say that the top one is the closest match isn't always true, and its
also important to distinguish between what is a real match versus what
is a spurious match.

\subsubsection{1.5. Where to go from here?}\label{where-to-go-from-here}

There is a lot of stuff in the NCBI website, so it is worth exploring.
Another very important sequence resource is the database of individual
genome projects. These can have the whole genome and/or the individual
sequence reads that were used to construct the genome. Very often they
have an online BLAST engine that you can search that database with.
There are also public databases that people have downloaded genome
sequence data onto that can be searched.

\subsection{2. Alignments}\label{alignments}

\subsubsection{2.1. Pairwise alignments}\label{pairwise-alignments}

\href{http://blast.ncbi.nlm.nih.gov/Blast.cgi?PAGE_TYPE=BlastSearch\&BLAST_SPEC=blast2seq\&LINK_LOC=align2seq}{Align}
aligns two sequences. In other words, this is a pairwise alignment tool.
This is one good way of aligning two sequences if they have a reasonable
match over their entire lengths. It has advantages and disadvantages. It
can align in both orientations, but it often struggles if the starts of
the sequence are not similar, or if the sequences don't match over their
entire lengths. It will also often produce very many short matches of
bits of the two sequences that are similar.

\begin{quote}
\emph{\textbf{Example 2}} 1. Copy the identifier for BRCA2 from human
(U43746.1) and mouse (U65594.1). 2. Paste them into the
\href{http://blast.ncbi.nlm.nih.gov/Blast.cgi?PAGE_TYPE=BlastSearch\&BLAST_SPEC=blast2seq\&LINK_LOC=align2seq}{align}
program. 3. Run it!
\end{quote}

\href{http://pipmaker.bx.psu.edu/cgi-bin/pipmaker?advanced}{Pipmaker}
can be used to very quickly find matches between two sequences is. Here
you can align sequences of very different lengths, and it also aligns in
both orientations, and is very fast. It produces a dot-plot type output
of the level of similarity. The disadvantage is that it doesn't give you
the actual sequence alignment, but it shows you what parts you can
align. You can also see repetitive elements in the dotplots.

\href{http://mobyle.pasteur.fr/cgi-bin/MobylePortal/portal.py?form=stretcher}{Stretcher}
is a useful tool for aligning long sequences that are quite similar. Its
usually OK if the sequences are not the same lengths, but not always.
You also have to make sure the sequences are in the same orientation. It
produces a nice aligned sequence output. It is part of the Mobyle
website that uses the EMBOSS suite of programs (run by the Pasteur).
This has quite a number of other, useful programs on there, such as
SQUIZZ for sequence/alignment conversions (see below).

\href{http://genome.lbl.gov/vista/index.shtml}{Vista} is another
alignment tool that produces a visual display of the level of similarity
across the alignment, allowing you to see conserved and diverged regions
easily. The aligner is surprisingly accurate and fast, and you can
include multiple sequences, although they all get pairwise aligned, not
multiply aligned as we will cover next .

\subsubsection{2.2. Multiple alignments}\label{multiple-alignments}

\href{https://www.ebi.ac.uk/Tools/msa/clustalw2/}{CLUSTALW2} is the most
commonly-used tool
{[}\href{http://www.ncbi.nlm.nih.gov/pubmed/17846036}{3}{]}. You can
find CLUSTALW all over the web. You must choose whether you do fast or
slow (depending on how accurate you want the results to be), and also
DNA or protein. There are also a few parameters that can be adjusted.
CLUSTALW2 often has trouble if the starts of the sequences are not very
similar, and can also have trouble with repeats and large indels.
Changing some of the parameters can help (often the gap settings).

\href{https://www.ebi.ac.uk/Tools/msa/clustalo/}{CLUSTAL OMEGA} is a
newer version of CLUSTALW (and will eventually replace it), which is not
yet as commonly used, but seems promising (Paper reference here
{[}\href{http://msb.embopress.org/content/7/1/539}{4}{]}).

\begin{quote}
\emph{\textbf{Example 3}} 1. Use this
\href{http://compbio.massey.ac.nz/wiki/data/c1/BRCA2_aa.fasta}{file} and
paste it into the interface (it contains the BRCA protein sequence from
7 species). 2. Run it with default parameters. 3. Look at the output. 4.
Highlight the alignments with colors. 5. Send the results to the
\href{https://www.ebi.ac.uk/Tools/phylogeny/clustalw2_phylogeny/}{EBI
Phylogeny}. 6. Look at the produced phylogentic tree. 7. What does it
tell us? 8. Do the lengths of the branches have a meaning?
\end{quote}

There are numerous other multiple alignment tools available, and
multiple alignments is a field of research in its own right. Many are
available as web-tools. Try out some and find one you like. But the most
important thing is that multiple alignments must be checked by eye for
accuracy. Therefore, you need to import the completed alignment into an
alignment editor, and then go through to check that nothing strange has
happened.

\subsection{3. Additional tools for sequence
analysis}\label{additional-tools-for-sequence-analysis}

There are a huge number of sequence analysis tools online, that do
useful things like make reverse complements of your DNA, translate it,
look for genes, look for repeat elements, make restriction maps, design
primers, etc. Try a few programs, and find something that works.

There are many different primer design tools out there, and I imagine
most do a good job. NCBI has one on their website called
\href{http://www.ncbi.nlm.nih.gov/tools/primer-blast/}{Primer-BLAST}.
Primer3Plus is one, which gives a large number of options, although is a
little hard to use at first.

For analysis of sequence data, small scale analyses are often done with
software that individual lab groups have. If not, BioEdit is a free
program for Windows that is a little painful but works OK. Geneious is a
(comparatively) cheap DNA analysis package that is useful for a wide
range of DNA analyses, such as alignments, phylogenetic tree-building
and cloning design that is powerful, easy to use, and also has the
ability to handle next-generation sequencing data (below). However
Geneious is not free software and as such costs money.

If you want to move into analysis of genome scale data, then you will
probably need to know some \emph{command-line} (below). For assembly of
Sanger sequences the
\href{http://www.phrap.org/phredphrapconsed.html}{Phred/Phrap/Consed}
package is a very powerful free assembly tool. I will deal with
next-generation sequence analysis below.

Attention! Often sequence/alignment formats are an issue when working
with different programs. Sometimes it is necessary to convert between
different sequence/alignment formats, and usually there are sites online
that can do this (e.g. \href{https://usegalaxy.org/}{GALAXY} see below).
Also be aware that Macs/\emph{Unix}/Windows use different
paragraph/space/newline notations, so sometimes these can be
incompatible between each other and need to be converted. Many cases
where you get errors for no apparent reason will be different line
return formats (in my experience!)

There are also increasing numbers of databases for a huge variety of
biological information. One good source for information about these is
the journal, \href{http://nar.oxfordjournals.org/}{Nucleic Acids
Research}. Periodically (once a year) they put out a special issue in
which each paper is a description of an online database of some sort. It
is a large issue! The DNA and protein databases are obviously the main
ones, and you are probably familiar with these. However, there are also
specialized database for model organisms (e.g.
\href{http://www.yeastgenome.org/}{Yeast SGD database}), types of genes,
and genomic comparisons.

Then there are more specialized ones, such as
\href{http://www.cbrc.jp/research/db/TFSEARCH.html}{TFSEARCH} that
searches the \href{http://jaspar.genereg.net/}{JASPAR database} public
version of the
\href{http://www.gene-regulation.com/pub/databases.html}{TRANSFAC
database} for known binding sites in your DNA (i.e.~a database for
finding cis-elements). If you want to discover new binding sites
\href{http://meme.nbcr.net/meme/}{MEME} might help.

You also might want to find gene ontology terms or pathways that are
associated to your gene-set
(\href{http://david.abcc.ncifcrf.gov/}{DAVID}) or your genomic regions
(\href{http://bejerano.stanford.edu/great/public/html/}{GREAT}).

You might want to identify protein-domains in our DNA or protein
sequences. Common databases and tools for such a task are
\href{http://www.ncbi.nlm.nih.gov/Structure/bwrpsb/bwrpsb.cgi?}{CD-SEARCH},
\href{http://prosite.expasy.org/}{PROSITE} or the
\href{http://pfam.xfam.org/}{PFAM} and
\href{http://supfam.org/SUPERFAMILY/hmm.html}{Superfamily} database.

Attention! Many of the specialized tools that are available with a
web-front-end impose limitations, e.g.~regarding sequence length, number
of motifs searched, etc. Generally speaking, the downloaded version of
these tools are the ones to circumvent such limitations. But then again,
we need to use the \emph{command-line} (see below).

\begin{quote}
\emph{\textbf{Example 4}} 1. Copy the proteins from this
\href{http://compbio.massey.ac.nz/wiki/data/c1/gene_set.txt}{file}. 2.
Paste the list to \href{http://david.abcc.ncifcrf.gov/}{DAVID}. 3. Look
at the functional annotation of the genes/proteins. 4. What are those
genes? 5. Look for Pathway enrichment
(\href{http://www.genome.jp/kegg/}{KEGG}) of the gene set and select one
pathway. 6. Explore the \href{http://www.genome.jp/kegg/}{KEGG}
database. 7. Convert the list using
\href{http://david.abcc.ncifcrf.gov/}{DAVID} to OFFICIAL\_GENE\_SYMBOLS.
8. Take the same gene and paste it into the
\href{http://string-db.org/}{STRING} search field. 9. Look at the
association network. What is the difference to
\href{http://www.genome.jp/kegg/}{KEGG}? 10. Take the first gene and
search it in the \href{http://genome.ucsc.edu/}{UCSC Genome Browser}.
\end{quote}

\begin{quote}
\emph{\textbf{Example 5}} 1. From the
\href{http://compbio.massey.ac.nz/wiki/data/c1/BRCA2_aa.fasta}{file},
take the human one and the chicken protein one and search for
protein-domains with \href{http://pfam.xfam.org/}{PFAM}. 2. What do you
expect to find? 3. What do you find??
\end{quote}

\subsection{4. Genome browsers}\label{genome-browsers}

Now we are in the age of sequencing, people often want to look at and
represent larger chunks of DNA, and also to integrate multiple pieces of
information into one view of your DNA of interest. If this is something
you are starting to do, then it might be worthwhile looking at using a
genome browser. Genome browsers are software tools that allow you to
view, in a scalable way, large pieces of DNA and to simultaneously view
many different features that are present in that sequence.

Usually you start by either putting your DNA sequence of interest into
the program , and then load various features onto this. For example, you
might want to load on where all the genes are, and then overlay this
with where histone variants map (from ChIP mapping) and which genes from
a related species are present. But you can basically put anything on it
(e.g.~I put primers in). In order to load features, you have to have a
special type of file, called a
\href{http://compbio.massey.ac.nz/wiki/\#!bioinf_files.md}{GFF3 file}.
This is actually not so special: it is just a tab-delimited text file
(you can make these with Excel) that contains the various information
you want to load on (gene name, position, orientation, etc). There are a
number of freely-available genome browsers you can use, such as the
Broad's \href{https://www.broadinstitute.org/igv/home}{IGV}.

The \href{http://genome.ucsc.edu/}{UCSC Genome Browser} is also of this
type, but here lots of data is already integrated and you can upload
specail annotation for the sequences in form of
\href{http://compbio.massey.ac.nz/wiki/\#!bioinf_files.md}{bed-files}.
This browser is actually a piece of software that one can use to create
one's own genome browser. Other projects use it as well, e.g.~the
\href{http://microbes.ucsc.edu/}{UCSC Microbial Genome browser}, etc.

\subsection{5. Next-generation sequence
analysis}\label{next-generation-sequence-analysis}

The costs of sequencing are dropping at a phenomenal rate, due to the
development of so-called next-generation sequencing technologies. Many
of these technologies produce a very large number of sequences of short
length that need to be assembled to get the original DNA sequence, or
mapped back to the genome (e.g.~in the case of mRNA sequencing). The
sheer numbers of sequences generated by these technologies means that in
many cases traditional tools for sequence analysis don't work.
More-and-more biologists have to learn to deal with high volumes of
sequences, and often this requires using the \emph{command-line} and
programming. If you are interested in continuing with biological
research, then gaining experience in dealing with this sort of data will
become an advantage for you. The biggest problem currently in biology is
the lack of people who can analyse large data sets, and can do so in a
quantitative manner. In most cases this requires the use of
\emph{command-line} driven software. Luckily, the majority are pretty
simple to use once you understand the basics. The great thing about
\emph{command-line} software is that almost all of it is free! It takes
a bit more time to get comfortable using it, and the support depends on
the software, but at least you don't have to harass your supervisor for
money!

This is an area that is developing incredibly rapidly, with a
bewildering range of tools that all seem to do the same or similar
things. Probably the best thing to do is to talk to someone who has done
these kinds of analyses before to help you get started. There are many
papers being published on a monthly basis providing new and varied ways
to a whole variety of analyses, but you don't want to spend your whole
time testing out new versions unless you have to. If in doubt, let
published studies doing the similar things to you guide you (if
possible).

There is also intensive development of downstream analyses of data
generated by these technologies. For transcriptome data, you may want to
know what set of genes has changed significantly between two different
samples (e.g.~a mutant and wild-type), and what classes these genes fall
into. Alternatively, you may want a list of where all the polymorphisms
from a genome sequence fall; what genes they fall into, etc. These kinds
of analyses are now often done using \emph{command-line} driven
software. An important feature to note is that next generation sequence
datasets are often publically available (usually a criterion for
publishing), therefore you can analyse these datasets yourself, which
means that you don't even have to do the experiments and then pay for
the sequencing yourself. This is important to keep in mind, as often the
people who originally produced the dataset did so for a completely
different reason, but the information that is important for you is in
there, and just needs to be analysed to get it out.

Another key aspect in the analysis of biological data is statistical
analysis. For basic statistics, there are commercial packages available
that will do statistics on data you enter. However, you should be
familiar with the statistical analyses that are being performed: how
they work and what they mean. As more-and-more data are produced, there
is a need to do more sophisticated analyses, and to be able to plot
these data. For this, more powerful statistical packages may be
required. Many of these also require use of the \emph{command-line}. A
particularly good (and free!) one is called
\href{http://www.r-project.org/}{R}. Once again it takes a bit to learn,
but will definitely be beneficial in the long-term!

Finally, the best way to learn computational biology is to talk to
people who have experience in the area and who can point you in the
right direction. Very often, the hardest part is knowing where to start.
Therefore ask around: there are a number of people in the Institute who
have experience with bioinformatics and computational biology who will
be able to help.

\subsection{6. Getting data}\label{getting-data}

There is lots of data already out there, and depending of what kind of
data you are looking for there are different databases and tools
available to get this data, e.g.

\begin{itemize}
\itemsep1pt\parskip0pt\parsep0pt
\item
  \href{http://genome.ucsc.edu/cgi-bin/hgTables?command=start}{UCSC
  Table Browser}
\item
  \href{http://www.biomart.org/biomart/martview/}{BioMart}
\item
  \href{http://www.ncbi.nlm.nih.gov/geo/}{Gene Expression Omnibus}
\item
  \href{http://www.ncbi.nlm.nih.gov/sra}{Short read archive}
\item
  \href{http://www.ncbi.nlm.nih.gov/snp/}{dbSNP}
\end{itemize}

\subsection{7. A word on the
command-line}\label{a-word-on-the-command-line}

Many of the analysis tools that are available on the web, plus a whole
lot that are not, are also available as downloadable programs to run
locally on your computer. One of the most common is the set of BLAST
scripts -- you can download them all. Many of these require (e.g.~the
BLAST series) require \emph{command-line} usage, although increasing
numbers are doing it through graphical user interfaces. The advantage
with running these programs off your own computer is they are often
faster, and you can often put in bigger/more sequences that on the web
(they often limit this so people don't take over the whole server with a
huge job). I highly recommend trying out \emph{command-line} -- you can
do a lot with just a little knowledge, and it opens up a whole lot of
stuff you couldn't otherwise use (in particular free stuff!).

This often requires use of a \emph{Unix}/\emph{Linux} computers
(luckily, Macs are basically \emph{Unix}, but Windows is not). An easy
way to try out a \emph{Linux} working environment is through an
\emph{Linux}
\href{https://help.ubuntu.com/community/LiveCD\#How-To_LiveCD_Ubuntu}{LiveCD}.
After you download the \emph{Linux} system and burn it onto a DVD, you
insert it into your computer drive and restart the computer. It will
boot your computer into a \emph{Linux} system that is running from the
CD, without any installation. The next time you restart your computer
without the CD inserted, your old system starts as normal without any
change.

There are good little tutorials on the web for the \emph{command-line}
stuff (a short one can be found
\href{http://compbio.massey.ac.nz/wiki/\#!comp_unix.md}{here}), but it
is also a good idea to ask someone who already knows some to teach: a
quick run through the key points will be enough to get you started.

\subsection{8. Beyond}\label{beyond}

If you really get into sequence analysis, then you will need to know
some programming! Amazingly, this is not as scary as it sounds. There
are a variety of languages you can use, but
\href{https://www.python.org/}{Python} is an easy one to start with. On
of the main advantages of Python is the presence of a library written
for bioinformatics tasks \href{http://www.biopython.org/}{BioPython}.
This is collection of what are called ``modules'' that people have
already written to do all sorts of sequence analyses and manipulations.
This means you don't have to write the program yourself, but you just
have to figure out how to use the provided module (which admittedly can
take some time as well). These modules are not really programs in their
own right: you have to write a program to use the modules to do things
that you want. If you learn how to use Python and write scripts, it can
be incredibly powerful! Another language that is widely used is
\href{http://www.perl.org/}{Perl}. For statistical analyses, the
language \href{http://www.r-project.org/}{R} is very widely used and
there are also a large number of modules that can be utilized for
biological analyses, known as
\href{http://www.bioconductor.org/}{Bioconductor}.

Hint! The good news for people wary of \emph{command-line} is a lot of
these tools are slowly being incorporated into web-based packages, such
as \href{http://usegalaxy.org}{GALAXY} or
\href{https://www.broadinstitute.org/cancer/software/genepattern/}{GenePattern},
that have a graphical user interface. These are useful, however,
understanding the underlying tools is important to make sure that your
results are as expected.

\subsection{9. Links}\label{links}

\begin{itemize}
\itemsep1pt\parskip0pt\parsep0pt
\item
  BLAST - \url{http://blast.ncbi.nlm.nih.gov/}
\item
  NCBI - \url{http://www.ncbi.nlm.nih.gov/}
\item
  NCBI Nucleotide - \url{http://www.ncbi.nlm.nih.gov/nuccore}
\item
  Stretcher -
  \url{http://mobyle.pasteur.fr/cgi-bin/MobylePortal/portal.py?form=stretcher}
\item
  Vista - \url{http://genome.lbl.gov/vista/index.shtml}
\item
  CLUSTALW2 - \url{https://www.ebi.ac.uk/Tools/msa/clustalw2/}
\item
  CLUSTAL OMEGA - \url{https://www.ebi.ac.uk/Tools/msa/clustalo/}
\item
  Primer-BLAST - \url{http://www.ncbi.nlm.nih.gov/tools/primer-blast/}
\item
  Phred/Phrap/Consed - \url{http://www.phrap.org/phredphrapconsed.html}
\item
  TFSEARCH - \url{http://www.cbrc.jp/research/db/TFSEARCH.html}
\item
  JASPAR database - \url{http://jaspar.genereg.net/}
\item
  TRANSFAC database -
  \url{http://www.gene-regulation.com/pub/databases.html}
\item
  MEME - \url{http://meme.nbcr.net/meme/}
\item
  DAVID - \url{http://david.abcc.ncifcrf.gov/}
\item
  GREAT - \url{http://bejerano.stanford.edu/great/public/html/}
\item
  STRING - \url{http://string-db.org/}
\item
  KEGG - \url{http://www.genome.jp/kegg/}
\item
  CD-SEARCH -
  \href{http://www.ncbi.nlm.nih.gov/Structure/bwrpsb/bwrpsb.cg\%20i?}{http://www.ncbi.nlm.nih.gov/Structure/bwrpsb/bwrpsb.cgi?}
\item
  PROSITE - \url{http://prosite.expasy.org/}
\item
  PFAM - \url{http://pfam.xfam.org/}
\item
  Superfamily - \url{http://supfam.org/SUPERFAMILY/hmm.html}
\item
  UCSC Genome Browser - \url{http://genome.ucsc.edu/}
\item
  IGV - \url{https://www.broadinstitute.org/igv/home}
\item
  UCSC Microbial Genome Browser - \url{http://microbes.ucsc.edu/}
\item
  UCSC Table Browser -
  \url{http://genome.ucsc.edu/cgi-bin/hgTables?command=start}
\item
  BioMart - \url{http://www.biomart.org/biomart/martview/}
\item
  Gene Expression Omnibus - \url{http://www.ncbi.nlm.nih.gov/geo/}
\item
  Short read archive - \url{http://www.ncbi.nlm.nih.gov/sra}
\item
  dbSNP - \url{http://www.ncbi.nlm.nih.gov/snp/}
\item
  Python - \url{https://www.python.org/}
\item
  BioPython - \url{http://www.biopython.org/}
\item
  Perl - \url{http://www.perl.org/}
\item
  R - \url{http://www.r-project.org/}
\item
  Bioconductor - \url{http://www.bioconductor.org/}
\item
  GALAXY - \url{https://usegalaxy.org/}
\item
  GenePattern -
  \url{https://www.broadinstitute.org/cancer/software/genepattern/}
\item
  Ubuntu LiveCD -
  \url{https://help.ubuntu.com/community/LiveCD\#How-To_LiveCD_Ubuntu}
\end{itemize}

\subsection{10. References}\label{references}

\begin{enumerate}
\def\labelenumi{\arabic{enumi}.}
\itemsep1pt\parskip0pt\parsep0pt
\item
  \href{http://www.cshlpress.com/default.tpl?cart=14004538673655488\&fromlink=T\&linkaction=full\&linksortby=oop_title\&--eqSKUdatarq=466}{Mount
  DM. Bioinformatics: Sequence and Genome Analysis (2nd ed.). Cold
  Spring Harbor Laboratory Press (2004): Cold Spring Harbor, NY. ISBN
  0-87969-608-7.}
\item
  \href{http://www.ncbi.nlm.nih.gov/pubmed/?term=9254694}{Altschul SF et
  al. Gaped BLAST and PSI-BLAST: a new generation of protein database
  search programs, Nucleic Acids Res (1997). 25:3389-3402.}
\item
  \href{http://www.ncbi.nlm.nih.gov/pubmed/17846036}{Larkin MA et al.
  Clustal W and Clustal X version 2.0. Bioinformatics (2007), 23,
  2947-2948.}
\item
  \href{http://msb.embopress.org/content/7/1/539}{Sievers F et al. Fast,
  scalable generation of high‐quality protein multiple sequence
  alignments using Clustal Omega.Mol Syst Biol. (2011) 7: 539. DOI:
  10.1038/msb.2011.75}
\end{enumerate}

\textbf{\emph{FILE: bioinf}seqintro.md - Download as
\href{http://compbio.massey.ac.nz/wiki/data/c1/doc/bioinf_seqintro.pdf}{PDF}
- Sebastian Schmeier - Last update: 2014/05/2014\_}

\end{document}
